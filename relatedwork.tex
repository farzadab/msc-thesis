\chapter{Related Work}
\label{ch:relatedwork}


Locomotion in humans and other animals is a long-standing problem. Different aspects of this problem have been the subject of study in numerous fields for decades, such as computer graphics, robotics, biomechanics, control, and more recently machine learning. In this work, however, we will be focusing on the results from computer graphics and to a more limited extent, the machine learning community.
% More specifically, this article is concerned with learning locomotion skills for physics-based character animation.



% One argument against kinematic methods is that directly manipulating object positions can lead to solutions which do not agree with the laws of physics, such as changing the center of mass of an object in mid-air. Therefore, the only way to get truly realistic and achievable motions is by using the laws of physics. This line of thinking leads us to physics-based methods. Then the question becomes, how to come up with the appropriate torques and forces to apply in order to get the desirable outcome.

\section{Motion Symmetry}
Motion symmetry has been a topic of interest for many years in the study of human motion and movement biomechanics. 
Symmetric motions are perceived to be more attractive, e.g., for dance~\cite{danceSymmetry},
and gait symmetry is seen as a desirable outcome for physiological manipulation~\cite{robinson1987use}.
While symmetry is a common assumption in the study of gait and posture, 
individual gaits often do exhibit asymmetries due to various possible functional causes~\cite{seelet}.
We refer the reader to a past review article~\cite{SADEGHI200034} for insights
into the degree of symmetry of lower limbs movement during able-bodied gait
and the potential influence of limb dominance on the motion symmetry of the lower extremities
and human gaits~\cite{riskowski}.
It is also not obvious how to best quantify 
the asymmetry of human gaits, and thus specific symmetry metrics have been
proposed~\cite{symmetry_measures,symmetry_phase_portrait}.

% \section{Energy-Efficient Locomotion}
% \red{TODO if time permits}


\section{Kinematic Locomotion}

Key-framing is a common approach to making characters move. In this technique the animator has to decide the character position and joint orientations for each frame, often using computer-assisted software. However, this approach is time-consuming and requires a highly skilled user. The use of motion capture systems is an alternative approach, but in the past, this has required expensive equipment and is constrained by the confines of the studio. Much work has been put into reusing captured motions using techniques such as Motion Graphs \cite{Kovar:2002:MG:566570.566605}. Recent approaches, such as Phase-Functioned Neural Networks  \cite{PFNN} and Mode-Adaptive Neural Networks \cite{MANN}, use neural networks to learn kinematic models. They can act based on a directional command from the user and can respond to terrain height variations.

Key-framing relies heavily on the mental biases and knowledge of the animator. It is therefore perhaps the simplest way of incorporating all forms of expert knowledge into the resulting motion, including energy efficiency and symmetry.
It is also common in kinematic-based approaches to mirror all the available motion data in order to double the effective size of the dataset and to reflect the often-symmetric nature of human locomotion, e.g.,~\cite{bruderlin,HoldenPFNN}.
% Furthermore, as kinematic methods are based on motion capture data, they are inherently biased towards the way that humans tend to move.




\section{Physics-Based Locomotion}

The robust control of physics-based character locomotion is a long-standing challenge
for character animation. We refer the reader to a survey paper for a detailed history~\cite{STAR2012}.  
An early and enduring approach to controller design has been to structure control policies around finite state
machines (FSMs) and feedback rules that use a simplified abstract model or feedback law.  These general
ideas have been applied to human athletics, running~\cite{Hodgins95}, and a rich variety of walking
styles~\cite{Yin07,Coros10,LeeYS10}. Many controllers developed for physics-based animation
further use optimization methods to improve controllers developed around an FSM-structure, or use an FSM to 
define phase-dependent objectives for an inverse dynamics optimization to be solved at each time step.
Policy search methods, e.g., stochastic local search or CMA~\cite{Hansen06}, can be used to optimize the 
parameters of the given control structures to achieve a richer variety of motions, e.g.,~\cite{Yin07,Coros11}, and
efficient muscle-driven locomotion~\cite{Wang09}.  
Many of the FSM controllers use hard-coded symmetries, which assign the roles of stance-leg or swing-leg
to the left and right legs, as a function of the FSM state.
% It is common in kinematic-based approaches to locomotion, e.g.,~\cite{bruderlin,HoldenPFNN}, 
% to also mirror all the available motion data
% in order to double the effective size of the data set, and to reflect the often-symmetric nature of human locomotion.
The trajectory optimization-based methods also commonly assume motion symmetry when convenient, e.g.,~\cite{majkowska2007flipping}.


More recently, locomotion synthesis has attracted significant
attention from the reinforcement learning (RL) community, where the OpenAI Gym tasks have 
become a popular RL benchmark~\cite{ref:OpenAI-Gym}. In this context, symmetry constraints are commonly 
not imposed and the torque limits are unrealistic. As a consequence, the resulting motions are often idiosyncratic and have noticeable asymmetries.
Further work extends these efforts in a variety of ways, including traversing challenging terrains~\cite{ref:deepmindParkour}.
More realistic and dynamic motions can be achieved with the help of motion-capture clips~\cite{2017-TOG-deepLoco,2018-TOG-deepMimic} and these use what in \Cref{ch:symmetry} is referred to as the PHASE symmetry method, with the goal of more efficient learning. \cite{2016-TOG-controlGraphs} uses a variation of PHASE in which individual strides (half steps) are mirrored and concatenated to generate symmetric reference motions.
However, there exist no robust documented experiments to verify efficiency gains.
The efficient learning of controllers that are capable of producing high-quality motion for realistic-strength characters remains
a challenging problem in the absence of motion capture data. Recent work makes
progress on this problem using RL with a combination of energy optimization, learning curriculum, and 
an auxiliary motion symmetry loss~\cite{Yu-SIGGRAPH-2018}, which we shall refer to as the LOSS method.
% 

%%%%%%%%%%%%% earlier version of physics-based
% One approach to this problem is to put all the requirements, including the initial and final pose, into a giant optimization problem constrained by the Newtonian laws and optimize a cost function such as energy consumption \cite{spacetime_constraints_luxo}. This optimization problem can then be solved by an existing algorithm such as the quadratic programming (QP) methods. However, the existence of discontinuities, such as contacts, can be problematic in this setting. One solution is to use soft constraints at the start of the optimization procedure and gradually make the constraints more rigid \cite{Mordatch:2012:DCB:2185520.2185539}.

% Another approach that is more suitable to dynamic settings like video games and robot control is that instead of learning one set of fixed actions the agent learns a controller. The controller can observe the target and optionally the state of the character in order to output the actions that achieve this target. \citeauthor{Tassa2014ControllimitedDD} usef classical control theory to learn a controller for a full 3D humanoid model. \citeauthor{2013-TOG-MuscleBasedBipeds} used an evolutionary algorithm to co-design the character morphology along with the final controller for muscle-based characters. The design of the controller is commonly posed as an optimization problem. However, SIMBICON \cite{Yin07} provides a meticulously designed framework that allows manual creation of controllers that work even across different simulators.

% Physics-based approaches can also make use of motion capture data. DeepMimic \cite{2018-TOG-deepMimic} allows motions to be learned simply from a short motion capture clip. The algorithm has been shown to work on different motions and character morphologies. DeepLoco \cite{2017-TOG-deepLoco} uses motion capture data to learn a low-level controller that is in turn controlled by a high-level controller that can plan the motion by sensing its surrounding environment.

% It is possible to get a walking or running gait without the use of motion capture data or imposing some form of structure on the solution. However, the resulting motions are usually far from appealing with an asymmetric gait and the hands flailing about \cite{emergence_locomotion}. 


% 
% 

% A recent result investigates how DRL problems can become prone to learning plateaus because
% of winner-take-all solution modes~\cite{ray_interf}. These can easily arise in DRL because the distribution of data
% for policy learning is directly influenced by the policy itself. Thus in the case of multiple diverging 
% decision paths, one of the modes will quickly dominate. The choice of whether to encourage symmetry,
% and how to do so, may create an optimization landscape that exhibits similar properties.



% \section{Curriculum Learning}
% \red{TODO if time permits}