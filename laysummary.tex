%% The following is a directive for TeXShop to indicate the main file
%%!TEX root = diss.tex

%% https://www.grad.ubc.ca/current-students/dissertation-thesis-preparation/preliminary-pages
%% 
%% LAY SUMMARY Effective May 2017, all theses and dissertations must
%% include a lay summary.  The lay or public summary explains the key
%% goals and contributions of the research/scholarly work in terms that
%% can be understood by the general public. It must not exceed 150
%% words in length.

\chapter{Lay Summary}

Reinforcement learning offers a flexible approach to learning locomotion skills in simulation.
This work investigates two approaches that can make the learned motions more realistic.
We incorporate gait symmetry into the learning process. We also propose to begin the learning process with exceptionally strong characters, which enables them to rapidly discover good solution modes, and then progressively revert to a weaker character in order to obtain a more realistic motion.

% a more appropriate way of enforcing torque limits on the simulated character.


% Using over-powered characters in simulation can help the learning. However, this can degrade the motion quality. We will propose a more appropriate way of enforcing torque limits on the simulated character without degrading the performance.

% On the other hand, able-bodied humans tend to have a fairly symmetric gait. We will explore methods that can incorporate this knowledge into the learning process.