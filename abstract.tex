%% The following is a directive for TeXShop to indicate the main file
%%!TEX root = diss.tex

\chapter{Abstract}


Deep reinforcement learning offers a flexible approach to learning physics-based locomotion. However, these methods are sample-inefficient and the result usually has poor motion quality when learned without the help of motion capture data.
This work investigates two approaches that can make motions more realistic while having equal or higher learning efficiency.

First, we propose a way of enforcing torque limits on the simulated character without degrading the performance. Torque limits indicate how strong a character is and therefore has implications on how realistic the resulting motion looks. We show that using realistic limits from the beginning can hinder training performance. Our method uses a curriculum learning approach in which the agent is gradually faced with more difficult tasks. This way the resulting motion becomes more realistic without sacrificing performance.


Second, we explore methods that can incorporate left-right symmetry into the learning process which highly increases the motion quality. Gait symmetry is an indicator of health and asymmetric motion is easily noticeable by human observers. We compare two novel approaches as well as two existing methods of incorporating symmetry in the reinforcement learning framework. We also introduce a new metric for evaluating gait symmetry and confirm that the resulting motion has higher motion quality.

% This document provides brief instructions for using the \class{ubcdiss}
% class to write a \acs{UBC}-conformant dissertation in \LaTeX.  This
% document is itself written using the \class{ubcdiss} class and is
% intended to serve as an example of writing a dissertation in \LaTeX.
% This document has embedded \acp{URL} and is intended to be viewed
% using a computer-based \ac{PDF} reader.

% Note: Abstracts should generally try to avoid using acronyms.

% Note: at \ac{UBC}, both the \ac{GPS} Ph.D. defence programme and the
% Library's online submission system restricts abstracts to 350
% words.

% Consider placing version information if you circulate multiple drafts
%\vfill
%\begin{center}
%\begin{sf}
%\fbox{Revision: \today}
%\end{sf}
%\end{center}


