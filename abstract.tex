%% The following is a directive for TeXShop to indicate the main file
%%!TEX root = diss.tex

\chapter{Abstract}


Deep reinforcement learning offers a flexible approach to learning physics-based locomotion. However, these methods are known to be sample-inefficient and the result usually has poor motion quality without the use of motion capture data. This work investigates two approaches that can make motions more realistic all the while having an equal or higher learning efficiency.

Using characters with high torque limits in simulation can speed up learning, however, this can degrade the motion quality. We will propose a more appropriate way of enforcing torque limits on the simulated character without degrading the performance. Furthermore, able-bodied humans tend to have a fairly symmetric gait. We will explore methods that can incorporate this knowledge into the learning process which highly increase the motion quality.


% This document provides brief instructions for using the \class{ubcdiss}
% class to write a \acs{UBC}-conformant dissertation in \LaTeX.  This
% document is itself written using the \class{ubcdiss} class and is
% intended to serve as an example of writing a dissertation in \LaTeX.
% This document has embedded \acp{URL} and is intended to be viewed
% using a computer-based \ac{PDF} reader.

% Note: Abstracts should generally try to avoid using acronyms.

% Note: at \ac{UBC}, both the \ac{GPS} Ph.D. defence programme and the
% Library's online submission system restricts abstracts to 350
% words.

% Consider placing version information if you circulate multiple drafts
%\vfill
%\begin{center}
%\begin{sf}
%\fbox{Revision: \today}
%\end{sf}
%\end{center}


