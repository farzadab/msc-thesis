%% The following is a directive for TeXShop to indicate the main file
%%!TEX root = diss.tex

\chapter{Abstract}

Deep reinforcement learning has been getting more and more attention lately for ... . However, these methods still cannot be applied to any problem simply out of the box. This motivates us to take a closer look at specific details of how these methods were conceived in order to understand the failure cases \red{(better?)} and its limits. Specifically we look into Policy Gradients, one class of Deep Reinforcement Learning algorithms, and our efforts lead us to new ways of interpreting how these algorithm work and their relationship with other similar approaches and it sheds light on how these methods can be combined together.

Next, we turn our attention to the set of tasks that we are more interested in, specifically locomotion, and investigate how we can setup our problems to make them more compatible with these methods. We look at task hyper-parameters that are usually neglected in the broader reinforcement learning community and investigate their effects.



% This document provides brief instructions for using the \class{ubcdiss}
% class to write a \acs{UBC}-conformant dissertation in \LaTeX.  This
% document is itself written using the \class{ubcdiss} class and is
% intended to serve as an example of writing a dissertation in \LaTeX.
% This document has embedded \acp{URL} and is intended to be viewed
% using a computer-based \ac{PDF} reader.

% Note: Abstracts should generally try to avoid using acronyms.

% Note: at \ac{UBC}, both the \ac{GPS} Ph.D. defence programme and the
% Library's online submission system restricts abstracts to 350
% words.

% Consider placing version information if you circulate multiple drafts
%\vfill
%\begin{center}
%\begin{sf}
%\fbox{Revision: \today}
%\end{sf}
%\end{center}


