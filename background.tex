\chapter{Background}
\label{ch:background}

\section{Reinforcement Learning}


\label{sec:background_rl}

\ac{RL} emerges from the idea that we humans tend to learn about the world through interaction. We can observe the world around us and act in certain ways in order to achieve our goals and at the same time learn more about the world that we live in. \ac{RL} is a computational approach to learning from interactions with the final goal of maximizing a numerical reward signal \cite{rlbook}.

This approach makes \ac{RL} applicable to a wide variety of application, but at the same time, this generality proves to be problematic as it yields many challenges. The learner is not told which actions are better or worse and it has to figure everything out itself using a weak reward signal. To make matters worse, the consequences of an action might not be immediately visible to the learner as they might be revealed only after a long duration of time.


\subsection{Markov Decision Process}

The problem formulation in \ac{RL} is based on the concept of a \ac{MDP}. The \ac{MDP} is defined by a tuple $\{\mathcal{S}, \mathcal{A}, P, r, \gamma\}$, where $S \in \mathbb R^n, A \in \mathbb R^m$ are the state space and action space of the problem, the transition function $P: S \times S \times A \to [0, \infty)$ 
is a probability density function, with $P(s_{t+1} \mid s_t, a_t)$ being the probability density of visiting $s_{t+1}$ given that at state $s_t$, the system takes action $a_t$.
The reward function $r: S\times A \to \mathbb R$ gives a scalar reward for each transition of the system. 
$\gamma \in [0, 1]$ is the discount factor. A deterministic \ac{MDP} is a special case where the transition function as well as the initial state distribution are deterministic.

Tasks can be categorized into two categories of \textit{episodic} and \textit{continuing} \cite{rlbook}. Episodic tasks can naturally be divided into subsequences known as \textit{episodes}, such as a single match in sports or one play of a game. Each episode ends after a certain period of time, known as the \textit{time horizon}, has passed or a pre-specified \textit{terminal state} has been reached. In continuing tasks, the task goes on without limit and there is no natural notion of an episode present. Here, we will consider the episodic case with a fixed time horizon $T$; for a more in depth discussion please refer to \cite{rlbook}. The goal of reinforcement learning is to find a parameterized policy $\pi_\theta$, where $\pi_\theta: S \times A \to [0, \infty)$ is the probability density of $a_t$ given $s_t$, that solves the following optimization problem:

\begin{align}
\label{eq:return}
\mathop{\mathrm{max}}_\theta J(\theta) = &\mathbb{E}_{\tau\sim p_\theta}\left[\sum_{t=0}^{T-1}\gamma^t{r({s}_t, {a}_t)} \right].
\end{align}

Here, $\tau = (s_0, a_0, r_0, \dots, s_T)$ is known as a \textit{trajectory} and the probability of encountering a trajectory is computed according to $p_\theta(\tau) = P(s_0) \Pi_{t=0}^{T-1} \pi_\theta(a_t|s_t) P(s_{t+1}|s_t,a_t)$. $\gamma \in [0,1]$ is called the discount factor and controls how much we care about future rewards rather than immediate rewards. The (discounted) sum of the rewards is also known as the \textit{return} of a trajectory:
\begin{align*}
    G(\tau) \doteq \sum_{t=0}^{T-1}\gamma^t{r({s}_t, {a}_t)}.
\end{align*}

Another important categorization of tasks is based on state and action spaces being discrete or continuous. In this article we will be focusing on tasks where both state and action spaces are continuous.

\section{Policy Gradient Methods}
\label{sec:background_pg}

Over the years many algorithms have been proposed for solving \ac{RL} tasks and each are useful in certain scenarios. This section will explain the \ac{PPO} algorithm which we will use in the following chapters. For a discussion of other existing methods please refer to \cite{rl_survey}.

\ac{PPO} belongs to the policy gradient methods class which have been show to work well on continuous tasks. The idea behind the \ac{PG} algorithm is pretty straight-forward. In short, it tries to optimize the average return by (approximately) computing its gradient and taking a gradient ascent step in order to increase it.
% However, doing this naively is not possible. and that is where the algorithm works its magic. Even with such a high-level description, we might be able to guess some of the problems that can arise, such as: how to find the global optima, and are the gradients well-behaved or not. We will get back to these problems soon enough, but let us see how the algorithm works first.

In order to optimize the objective \ac{PG} directly optimizes the policy $\pi$. One of the underlying assumption of \ac{PG} is that the policy should be stochastic rather than deterministic for this algorithm to work, though this assumption can be relaxed \cite{ddpg}. Furthermore, we assume that this stochastic policy is parametrized by parameters $\theta$ and therefore the algorithm's job is to find the optimal parameters $\theta^*$ that maximize \Cref{eq:return}. Now, in order to maximize the objective, we need to know the gradient $\nabla_\theta J(\theta)$. Since computing this gradient requires knowledge about the dynamics of the \ac{MDP}, \ac{PG} approximates it by using the REINFORCE trick \cite{williams1992simple}:

\begin{align}
    \label{eq:obj_grad}
    \nabla_\theta J(\theta) &= \nabla_\theta \int G(\tau) p_\theta(\tau) \\
    &= \int G(\tau) \nabla_\theta p_\theta(\tau)\\
    &= \int G(\tau) p_\theta(\tau) \nabla_\theta \log p_\theta(\tau)\\
    &= \mathbb{E}_{\tau \sim p_\theta} \left[ G(\tau) \nabla_\theta \log p_\theta(\tau) \right]
\end{align}

We can switch the integral and the gradient operator under some conditions that we will explore later on. The equality is based on the following identity:

\begin{align*}
    p_\theta(\tau)\nabla_\theta \log p_\theta(\tau) = p_\theta(\tau) \frac{\nabla_\theta p_\theta(\tau)}{p_\theta(\tau)} = \nabla_\theta p_\theta(\tau)
\end{align*}

Next, we can expand $\nabla_\theta \log p_\theta$:

\begin{align}
    \nabla_\theta \log p_\theta &= \nabla_\theta \left[ \log p(s_1) + \sum_{t=1}^T \log \pi_\theta(a_t|s_t) + \log p(s_{t+1} | s_t, a_t) \right]\\
    &= \nabla_\theta \left[ \sum_{t=1}^T \log \pi_\theta(a_t|s_t) \right]\\
    &= \sum_{t=1}^T \nabla_\theta \log \pi_\theta(a_t|s_t)
\end{align}

Substituting this back into \Cref{eq:obj_grad} we arrive at the following expression. For simplicity the discount factor, $\gamma$, has been set to one:

\begin{align}
    \nabla_\theta J(\theta) &= \mathbb{E}_{\tau \sim p_\theta} \left[ G(\tau) \sum_{t=1}^T \nabla_\theta \log \pi_\theta(a_t|s_t) \right]\\
    &= \mathbb{E}_{\tau \sim p_\theta} \left[ \left( \sum_{t=1}^T r_t \right) \left( \sum_{t=1}^T \nabla_\theta \log \pi_\theta(a_t|s_t) \right) \right]
\end{align}

Using this formulation we can use Monte Carlo sampling \red{cite} to approximately compute the gradient in order to iteratively improve the policy:


\begin{align}
    \nabla_\theta J(\theta) &\approx \frac{1}{N} \sum_{i=1}^{N} \left[ \left( \sum_{t=1}^T r_{i,t} \right) \left( \sum_{t=1}^T \nabla_\theta \log \pi_\theta(a_{i,t}|s_{i,t}) \right) \right]
\end{align}

Where, $\tau_i = (s_{i,1}, a_{i,1}, r_{i,1}, \dots, s_{i,T})$ are sample trajectories from $p_\theta$. With this we arrive at the REINFORCE algorithm \cite{williams1992simple}.

Unfortunately, this is not a good estimator in practice as its variance can be quite high. There are multiple tricks that try to alleviate this problem. The simplest is to increase the number of sampled trajectories $N$, however, this also makes the algorithm less efficient. One observation is that a reward time step $t$ only causally depends on actions that were made until time $t$ and are independent from decisions that are made afterwards, or in other words, action $a_t$ can only be responsible for the cost to go from time $t$ forward. With some abuse of notation we can write:
\begin{align}
    \nabla_\theta J(\theta) &\approx \mathbb{E}_{\tau \sim p_\theta} \left[ \sum_{t=1}^T \left( \nabla_\theta \log \pi_\theta(a_t|s_t)  \sum_{t'=t}^T r_t \right) \right]\\
    &= \mathbb{E}_{s_t \sim p_\theta} \left[T \nabla_\theta \log \pi_\theta(a_t|s_t)  \sum_{t'=t}^T r_t \right],
\end{align}
where $s_t$ is any state randomly sampled from a randomly sampled trajectory.

Another trick is to use a learned value function $\hat{V}(s)$ also known as a \textit{critic}. It can be shown that subtracting out a fixed value from the cumulative return in the above formula does not change the value of the expectation. Therefore, if $\hat{V}(s_t)$ is a good estimate of $\sum_{t'=t}^T r_t$ then the following approximator would have lower variance:
\begin{align}
    \nabla_\theta J(\theta) &\approx \mathbb{E}_{s_t \sim p_\theta} \left[T \nabla_\theta \log \pi_\theta(a_t|s_t) \left( \sum_{t'=t}^T r_t - \hat{V}(s_t) \right) \right].
\end{align}